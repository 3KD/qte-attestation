\documentclass[11pt]{article}
\usepackage{amsmath,amssymb,amsthm}

\title{E1 -- NVADE--SEC Encryption Game (Baseline T1)}
\author{QTE / NVADE--SEC}
\date{}

\begin{document}
\maketitle

\section*{Preliminaries}

Fix $m \in \mathbb{N}$ and the message space
\[
  \mathcal{M}_m \coloneqq \{0,1\}^m,
\]
equipped with the uniform distribution when sampling random messages.
Let $\mathcal{K}_m \coloneqq \{0,1\}^m$ denote the key space for length-$m$
bitstrings.

We assume a NVADE key bundle
\[
  \mathsf{Bundle} \coloneqq
  \bigl(
    \mathsf{Family},\;
    \mathsf{Key},\;
    \mathsf{Chan},\;
    \mathsf{Threat}
  \bigr)
\]
as defined in E0. Here:
\begin{itemize}
  \item $\mathsf{Family}$ is the public NVADE encoder family
        (family id, series id, truncation, weighting, index order, norm),
  \item $\mathsf{Key}$ is a secret parameter choice $\theta$ together with a
        public key label, internal to the bundle,
  \item $\mathsf{Chan}$ specifies an honest backend/channel implementation,
  \item $\mathsf{Threat}$ encodes the adversary's observational power.
\end{itemize}

For the E1 baseline we only use the $\mathsf{Family}$ and the \emph{key label}
as a source of deterministic key material; no quantum hardware is involved.

\section*{Key derivation from the NVADE bundle}

Let $\mathrm{family\_id}$ and $\mathrm{key\_label}$ denote the public family
identifier and key label from the bundle. We form the string
\[
  s \coloneqq \texttt{family\_id} \,\|\, \texttt{"|"} \,\|\, \texttt{key\_label}
\]
and interpret its UTF-8 bytes as a sequence of unsigned integers
$(b_1,\dots,b_L)$. Define a 32-bit seed
\[
  \mathrm{seed} \coloneqq \Bigl( \sum_{i=1}^L b_i \Bigr) \bmod 2^{32}.
\]

This seed is then fed to a fixed pseudorandom generator (in implementation
terms, a numpy RNG) to obtain an $m$-bit key
\[
  K \in \mathcal{K}_m,
\]
with $K$ a deterministic function of the bundle's public identifiers. For the
E1--T1 baseline we regard this as an abstract deterministic key-derivation
function
\[
  \mathrm{KDF} : \mathsf{Bundle}^{\mathrm{pub}} \times \{m\} \to \mathcal{K}_m.
\]

\section*{Encryption and decryption}

Given $K \in \mathcal{K}_m$, encryption and decryption are defined by the
XOR-based one-time-pad rules:
\begin{align*}
  \mathrm{Enc}_K : \mathcal{M}_m &\to \mathcal{C}_m, &
  \mathrm{Enc}_K(M) &\coloneqq M \oplus K, \\
  \mathrm{Dec}_K : \mathcal{C}_m &\to \mathcal{M}_m, &
  \mathrm{Dec}_K(C) &\coloneqq C \oplus K,
\end{align*}
where $\mathcal{C}_m = \{0,1\}^m$ is the ciphertext space and $\oplus$ denotes
bitwise XOR.

For any fixed key $K$, this satisfies
\[
  \mathrm{Dec}_K(\mathrm{Enc}_K(M)) = M
  \qquad\text{for all } M \in \mathcal{M}_m,
\]
so the honest decryptor can always recover the message in the absence of
implementation errors.

In the E1 baseline we treat the NVADE bundle as providing $K$ via the KDF
described above, but do not otherwise use the quantum structure of the family.
The goal is simply to anchor a concrete encryption game to the NVADE key
bundle API.

\section*{E1--T1 encryption game (keyed vs no-key adversary)}

We now define the baseline test T1 for a fixed $m$ and fixed bundle.

\subsection*{Experiment}

Let $K \gets \mathrm{KDF}(\mathsf{Bundle}^{\mathrm{pub}}, m)$ be the derived
key. In each trial we perform:
\begin{enumerate}
  \item Sample a message $M \gets \mathcal{M}_m$ uniformly at random.
  \item Compute the ciphertext $C \coloneqq \mathrm{Enc}_K(M)$.
  \item \emph{Honest decryptor:} output
        $M_{\mathrm{hon}} \coloneqq \mathrm{Dec}_K(C)$.
  \item \emph{No-key adversary:} output
        $M_{\mathrm{adv}} \gets \mathcal{M}_m$ as a uniformly random guess,
        independent of $C$.
  \item Record the indicator variables
        \[
          S_{\mathrm{hon}} \coloneqq \mathbf{1}[M_{\mathrm{hon}} = M], \qquad
          S_{\mathrm{adv}} \coloneqq \mathbf{1}[M_{\mathrm{adv}} = M].
        \]
\end{enumerate}

Repeating this process for $N$ independent trials yields empirical success
rates
\[
  \hat{p}_{\mathrm{hon}} \coloneqq
    \frac{1}{N} \sum_{i=1}^N S_{\mathrm{hon}}^{(i)}, \qquad
  \hat{p}_{\mathrm{adv}} \coloneqq
    \frac{1}{N} \sum_{i=1}^N S_{\mathrm{adv}}^{(i)}.
\]

\subsection*{Theoretical baseline}

By construction,
\[
  S_{\mathrm{hon}} = 1 \quad\text{almost surely,}
\]
so in the idealised model the honest decryptor's success probability is
\[
  p_{\mathrm{hon}} = 1.
\]

For the no-key adversary, each guess is independent of $M$ and uniformly
distributed on $\mathcal{M}_m$, hence
\[
  p_{\mathrm{adv}} = \Pr[M_{\mathrm{adv}} = M]
  = \frac{1}{2^m}.
\]

Thus the E1--T1 baseline experiment should empirically produce
\[
  \hat{p}_{\mathrm{hon}} \approx 1,
  \qquad
  \hat{p}_{\mathrm{adv}} \approx 2^{-m},
\]
up to sampling noise. This is the behaviour confirmed by the current QTE
implementation for $m \in \{3,6,7,8\}$ with $N$ in the range $5\cdot 10^3$ to
$10^4$.

\section*{Role within NVADE--SEC}

The E1--T1 game does not claim any novel cryptographic hardness; it is a
deliberately simple construction that:

\begin{itemize}
  \item pins down a precise notion of ``encryption'' derived from a NVADE key
        bundle, using only the existing E0 interface,
  \item yields a clean, reproducible baseline gap between a keyed decryptor
        and a no-key adversary, with success scaling as $2^{-m}$,
  \item provides a concrete experimental object (E1) that can later be
        tightened or replaced by more sophisticated games using the same bundle
        abstraction and attestation machinery.
\end{itemize}

Subsequent tests (e.g.\ E1--T2) will depart from this idealised one-time-pad
setting by exposing structured transcripts or partial information and asking
whether an adversary can beat the $2^{-m}$ baseline, thereby probing more
subtle properties of NVADE-based keys and their attestation.
\end{document}
