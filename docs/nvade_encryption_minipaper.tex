\documentclass[11pt]{article}

\usepackage[a4paper,margin=1in]{geometry}
\usepackage{amsmath,amssymb,amsfonts}
\usepackage{graphicx}
\usepackage{hyperref}
\usepackage{booktabs}
\usepackage{siunitx}
\usepackage{authblk}

\newcommand{\ket}[1]{\left|#1\right\rangle}
\newcommand{\bra}[1]{\left\langle#1\right|}
\newcommand{\braket}[2]{\left\langle #1 \middle| #2 \right\rangle}

\sisetup{
  round-mode=places,
  round-precision=4
}

\title{Amplitude-Key Encryption and Trapdoor Witnesses from\\
Normalized Vector Amplitude Distribution Encoding (NVADE)}

\author[1]{Author Name}
\affil[1]{Affiliation, City, Country}
\date{\today}

\begin{document}

\maketitle

\begin{abstract}
We consider mathematical objects---in particular truncated series and finite vectors---as
primary quantum data, and encode them directly into normalized amplitude vectors via
Normalized Vector Amplitude Distribution Encoding (NVADE).
A NVADE object is a statevector
$\ket{\psi} = \sum_{k=0}^{N-1} \psi_k \ket{k}$ with $\sum_k |\psi_k|^2 = 1$,
where the coefficients $\psi_k$ are derived from a structured series and accompanied by
hash-verified metadata.
On top of this encoding we build three simple, reproducible primitives:

(1) a trapdoor witness test (Unit U31) that distinguishes a reference NVADE state from
an impostor via likelihood scores and ROC/AUC metrics, run both on a simulator and on an
IBM quantum device;

(2) a baseline classical encryption experiment (E1--T1) that measures honest-vs-adversary
success probabilities as a function of message length;

(3) a NVADE amplitude-key phase-mask primitive (E1--T2) where a NVADE state acts as a
symmetric key under a diagonal phase operator, achieving unit fidelity (up to numerical
precision) over thousands of trials.

All experiments emit machine-checkable U58 attestation cards that package counts,
metrics, and provenance with cryptographic hashes.  This note reports the concrete
numbers and their experimental setup, and is intended as a minimal, self-contained
reference for subsequent, more elaborate studies.
\end{abstract}

\section{Introduction}
\label{sec:intro}

A central theme in this work is to treat \emph{mathematical structure} as the payload
of a quantum state, rather than as an afterthought on top of generic qubit registers.
Normalized Vector Amplitude Distribution Encoding (NVADE) provides a deterministic map
from an ordered list of complex coefficients to a legal quantum statevector, together
with a metadata and hashing scheme that makes these objects portable across simulators
and hardware.

Let $t \in \mathbb{C}^L$ be the truncated coefficient sequence of some series or
structured object, with a fixed index ordering.  NVADE pads or truncates $t$ to a target
dimension $N$ (often $N = 2^n$ for an $n$-qubit system),
\begin{equation}
  w_k =
  \begin{cases}
    t_k, & 0 \le k < \min\{L,N\},\\
    0,   & \text{otherwise},
  \end{cases}
\end{equation}
and normalizes
\begin{equation}
  \ket{\psi} = \frac{1}{\|w\|_2}\sum_{k=0}^{N-1} w_k \ket{k}.
\end{equation}
The resulting statevector $\psi \in \mathbb{C}^N$ is accompanied by metadata describing
the source object and a cryptographic hash of a rounded representation of $\psi$.

On top of this backbone we instantiate three concrete experiments:

\begin{itemize}
  \item a \emph{trapdoor witness} (U31) that compares a reference NVADE state to an
        impostor via a scalar score and ROC/AUC analysis;
  \item a \emph{baseline encryption} experiment (E1--T1) which measures honest-vs-adversary
        decoding success as message length increases;
  \item a \emph{NVADE amplitude-key phase-mask} (E1--T2) which uses a NVADE state as a
        symmetric key under a diagonal phase operator, and tests the decryption fidelity.
\end{itemize}

Each run is recorded as a JSON object with counts, metrics, and provenance, and then
lifted into a U58 attestation card: a signed, hash-verified evidence object that can be
archived, rechecked, or used as input for downstream analyses.

\section{Methods (brief)}
\label{sec:methods}

We collect only the minimal definitions needed to make the results in
Section~\ref{sec:results} interpretable.  Full implementation details live in the
associated repositories.

\subsection{Trapdoor witness (U31)}
\label{sec:methods-u31}

Let $\ket{\psi_{\mathrm{ref}}}$ be a reference NVADE state and
$\ket{\psi_{\mathrm{alt}}}$ an alternative state on the same $n$-qubit space.
Measuring in the computational basis yields distributions
\begin{equation}
  p_{\mathrm{ref}}(x) = |\braket{x}{\psi_{\mathrm{ref}}}|^2, \quad
  p_{\mathrm{alt}}(x) = |\braket{x}{\psi_{\mathrm{alt}}}|^2,\quad x\in\{0,1\}^n.
\end{equation}
Given counts $\{c_{\mathrm{ref}}(x)\}$ and $\{c_{\mathrm{alt}}(x)\}$ from $S$ shots each,
we form empirical probabilities $\hat p$ and a (regularized) log-likelihood ratio score
\begin{equation}
  s(x) = \log\frac{\hat p_{\mathrm{ref}}(x) + \epsilon}{\hat p_{\mathrm{alt}}(x) + \epsilon},
\end{equation}
with a small $\epsilon > 0$ to avoid division by zero.
Sampling $s(x)$ under $p_{\mathrm{ref}}$ and $p_{\mathrm{alt}}$ gives ``honest'' and
``impostor'' score distributions; sweeping a threshold $\tau$ yields a receiver operating
characteristic (ROC) curve and an empirical area under the curve (AUC).

In Unit U31 we implement this test as a runner that:
\begin{enumerate}
  \item loads $\ket{\psi_{\mathrm{ref}}}$ and $\ket{\psi_{\mathrm{alt}}}$ from disk;
  \item prepares each state via an \texttt{initialize} gate and measures in the $Z$ basis;
  \item collects counts from both an ideal simulator and a hardware backend;
  \item computes ROC curves and AUC/TPR@1\%FPR metrics;
  \item writes these along with provenance into a JSON payload.
\end{enumerate}

\subsection{Baseline encryption (E1--T1)}
\label{sec:methods-e1t1}

E1--T1 is a classical baseline experiment that uses NVADE-derived data to parameterize a
simple symmetric-key encryption scheme on $m$-bit messages.

For each trial:
\begin{enumerate}
  \item Sample a uniformly random message $M \in \{0,1\}^m$ and a secret key $K$
        derived from a NVADE object.
  \item Form a ciphertext $C = E_K(M)$ by applying a keyed transformation determined
        by the NVADE-derived parameters.
  \item The honest receiver, with access to $K$, computes $\hat M_{\mathrm{honest}} = D_K(C)$.
  \item A blind adversary, without $K$, produces a guess $\hat M_{\mathrm{adv}}$ using only
        public information.
\end{enumerate}
We record the empirical honest and adversarial success rates over $N_{\mathrm{trials}}$
repetitions:
\begin{equation}
  \Pr_{\mathrm{honest}} = \Pr[\hat M_{\mathrm{honest}} = M], \quad
  \Pr_{\mathrm{adv}}    = \Pr[\hat M_{\mathrm{adv}} = M].
\end{equation}
The exact form of $E_K$ and $D_K$ in this baseline is intentionally simple; the goal is
to quantify the separation between a keyed decoder and a blind guesser as a function of
message length.

\subsection{Amplitude-key phase-mask primitive (E1--T2)}
\label{sec:methods-e1t2}

In E1--T2 the NVADE state itself acts as a symmetric key.  Let
\begin{equation}
  \ket{\psi_K} = \sum_{k=0}^{N-1} \psi_k \ket{k}
\end{equation}
be a fixed key state on $n$ qubits ($N=2^n$), and let $m \in \{0,1\}^n$ be a classical
message.
We define a diagonal phase-mask operator
\begin{equation}
  U_m = \bigotimes_{j=0}^{n-1} Z_j^{m_j},
\end{equation}
so that $U_m$ applies a Pauli-$Z$ on qubit $j$ when $m_j = 1$ and the identity otherwise.
Because $Z_j^2 = I$, we have $U_m^2 = I$.

Encryption is
\begin{equation}
  \ket{\psi_{\mathrm{enc}}(m)} = U_m \ket{\psi_K},
\end{equation}
and decryption applies the same operator again:
\begin{equation}
  \ket{\psi_{\mathrm{dec}}(m)} = U_m \ket{\psi_{\mathrm{enc}}(m)} = U_m^2 \ket{\psi_K}
  = \ket{\psi_K}.
\end{equation}
In E1--T2 we do not attempt a full cryptographic security analysis; the focus is on
implementing this primitive faithfully and measuring the decryption fidelity
\begin{equation}
  F(m) = |\braket{\psi_K}{\psi_{\mathrm{dec}}(m)}|^2
\end{equation}
across many randomly chosen messages.

\subsection{U58 attestation cards}
\label{sec:methods-u58}

Every experiment above emits a JSON run record with counts, metrics, and provenance.
A U58 attestation card is a higher-level evidence object that wraps one such run with:
\begin{itemize}
  \item a canonical serialization of the payload,
  \item a SHA256 hash of that serialization,
  \item a small header identifying the unit, test type, and backend,
  \item optional timestamps and human-readable notes.
\end{itemize}
These cards can be archived, signed, or published as the basic evidence units for
subsequent analyses, including any future noise-window advantage studies.

\section{Results}
\label{sec:results}

\subsection{Trapdoor witness on simulator and hardware (U31)}
\label{sec:results-u31}

We instantiated the U31 trapdoor witness on a three-qubit NVADE state derived from a
structured series (the ``prod3'' family) and a chosen impostor state on the same
Hilbert space.  For each backend we prepared both the honest and impostor states,
measured in the computational ($Z$) basis for a fixed number of shots, and computed
log-likelihood ratio scores as in Section~\ref{sec:methods-u31}.

Figure~\ref{fig:u31-roc} (not shown here) conceptually plots the ROC curves for two
tests, T1 (``prod3'') and T2 (``wrong-key-LLR''), on both an ideal simulator
(\texttt{aer\_simulator}) and the IBM \texttt{ibm\_torino} backend.
Table~\ref{tab:u31-metrics} summarizes the key numerical metrics.

\begin{table}[h]
  \centering
  \caption{U31 trapdoor witness metrics on simulator and hardware.  AUC is the empirical
  area under the ROC curve; TPR@1\%FPR is the true positive rate at a fixed false positive
  rate of $1\%$.}
  \label{tab:u31-metrics}
  \begin{tabular}{llllS[round-precision=4]S[round-precision=4]}
    \toprule
    Test & Backend kind & Backend name & Metric & {AUC} & {TPR@1\%FPR} \\
    \midrule
    T1 (prod3) & aer & aer\_simulator   &        & 0.6783105731 & 0.1936035156 \\
               & ibm & ibm\_torino      &        & 0.5414723158 & 0.2814941406 \\
    \midrule
    T2 (wrong-key-LLR) & aer & aer\_simulator &  & 0.5944366455 & 0.0656738281 \\
                       & ibm & ibm\_torino    &  & 0.6177199483 & 0.0917968750 \\
    \bottomrule
  \end{tabular}
\end{table}

All four runs emit U58 attestation cards with hashes that match on recomputation,
indicating that the stored payloads are self-consistent.  Qualitatively, the AUC values
are modestly above $0.5$ in all cases, demonstrating that the trapdoor witness is
non-trivial, and the hardware-induced noise shifts the ROC geometry without erasing
discriminability.

\subsection{Baseline encryption success rates (E1--T1)}
\label{sec:results-e1t1}

The E1--T1 baseline tests the separation between an honest decoder with access to a
NVADE-derived key and a blind adversary, as a function of message length $m$.
For each $m$ we run $N_{\mathrm{trials}}$ independent encryptions and record the
empirical success probabilities.

We carried out experiments for $m \in \{3,6,7,8\}$ with trial counts
$N_{\mathrm{trials}} \in \{5000, 10\,000\}$.  The results are summarized in
Table~\ref{tab:e1t1}.

\begin{table}[h]
  \centering
  \caption{E1--T1 baseline: honest and adversary success rates as a function of
  message length $m$.  Each row corresponds to one JSON run file produced by
  \texttt{e1\_t1\_encryption\_baseline.py}.}
  \label{tab:e1t1}
  \begin{tabular}{rrrrS[round-precision=4]S[round-precision=4]}
    \toprule
    $m$ & $N_{\mathrm{trials}}$ & & & {Honest success} & {Adversary success} \\
    \midrule
    3 &  5000 & & & 1.0000 & 0.1262 \\
    6 & 10000 & & & 1.0000 & 0.0152 \\
    7 & 10000 & & & 1.0000 & 0.0087 \\
    8 & 10000 & & & 1.0000 & 0.0043 \\
    \bottomrule
  \end{tabular}
\end{table}

For all configurations, the honest decoder recovers the message with probability
essentially equal to $1$ within sampling error, while the adversary's success probability
drops rapidly as $m$ increases, approaching the uniform-guess baseline $2^{-m}$.
Although this is only a simplified test and not a full cryptographic security proof,
it provides a concrete capacity-style plot: increasing the message length under this
scheme suppresses blind guessing advantage while preserving perfect decodability for
the keyed receiver.

These runs can also be wrapped in U58 cards, but in the present work we mainly treat
them as sanity checks for the more structured amplitude-key primitive in E1--T2.

\subsection{Amplitude-key phase-mask fidelity (E1--T2)}
\label{sec:results-e1t2}

In E1--T2 we implement the amplitude-key phase-mask primitive from
Section~\ref{sec:methods-e1t2} and test decryption fidelity.  The key is a six-qubit
NVADE state derived from a Ramanujan-type series, yielding $N = 2^6 = 64$ amplitudes.
We choose $m = 6$ message bits so that each bit controls a $Z$ on one of the six qubits.

For each trial, we:
\begin{enumerate}
  \item sample a random message $m \in \{0,1\}^6$,
  \item apply $U_m$ to $\ket{\psi_K}$ to obtain $\ket{\psi_{\mathrm{enc}}(m)}$,
  \item apply $U_m$ again to obtain the decrypted state $\ket{\psi_{\mathrm{dec}}(m)}$,
  \item compute the fidelity
        $F(m) = |\braket{\psi_K}{\psi_{\mathrm{dec}}(m)}|^2$
        by comparing the numerically known statevectors.
\end{enumerate}

Over $N_{\mathrm{trials}} = 5000$ trials we obtain:
\begin{align}
  \text{average fidelity} &\approx 1.0000000000000002, \\
  \text{minimum fidelity} &\approx 1.0000000000000004,
\end{align}
where the slight excursions above $1$ are attributable to floating-point roundoff.
Within numerical precision, the decryption fidelity is identically $1$ for all sampled
messages.

This confirms that the amplitude-key phase-mask construction behaves as expected:
given a fixed NVADE key state and a diagonal phase operator $U_m$ with $U_m^2 = I$,
a single application of $U_m$ encrypts and a second application perfectly decrypts,
without degrading the underlying key state in the idealized model used here.

In future work this primitive can be extended along several axes:
\begin{itemize}
  \item replacing the numerical inner-product fidelity with experimental estimates on
        real hardware,
  \item adding noise and hardware constraints to study robustness,
  \item layering this diagonal primitive with more general keyed unitaries and error
        correction or authentication mechanisms.
\end{itemize}

\section{Discussion and Outlook}
\label{sec:discussion}

The three experiments reported here---U31 trapdoor witnesses on simulator and hardware,
E1--T1 baseline encryption, and E1--T2 amplitude-key phase-masks---constitute a minimal
but concrete demonstration that NVADE-encoded series states can be used both as
\emph{objects of study} (via trapdoor witnesses and ROC analysis) and as
\emph{cryptographic resources} (via amplitude-key operations), with reproducible,
hash-verified evidence.

The U58 attestation cards provide a portable, machine-checkable representation of each
run, enabling downstream analyses such as noise-window advantage searches, multi-backend
comparisons, and long-term drift studies without re-running the raw experiments.

Subsequent papers can enrich this skeleton in several directions:
\begin{itemize}
  \item a full security definition and analysis for amplitude-key encryption, including
        indistinguishability games and adversary models;
  \item explicit noise models and hardware experiments, especially for devices where
        phase errors dominate;
  \item application-specific case studies (e.g.\ ergodic-theoretic signals, quantum
        chemistry spectra, or complex analytic functions) where the structure of the
        NVADE key interacts non-trivially with the modulation and verification stack.
\end{itemize}

\end{document}

