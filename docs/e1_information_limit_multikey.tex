\documentclass[11pt]{article}

\usepackage[margin=1in]{geometry}
\usepackage{amsmath,amssymb,amsthm}
\usepackage{bm}
\usepackage{hyperref}

\newtheorem{theorem}{Theorem}
\newtheorem{definition}{Definition}
\newtheorem{proposition}{Proposition}
\newtheorem{corollary}{Corollary}

\title{NVADE--SEC Information Limit and Multi--Key Modulation Channels}
\author{E.~Dollarhide}
\date{\today}

\begin{document}
\maketitle

\begin{abstract}
Normalized Vector Amplitude Distribution Encoding (NVADE) maps a structured
classical object---typically a truncated series---into a normalized state
vector $\bm{\psi} \in \mathbb{C}^N$.
In NVADE--SEC this state is then prepared on $n$ physical qubits
($N = 2^n$) and used as the substrate for encryption, attestation, and
randomness certification.
This note formalizes the information-theoretic limit of such channels
and extends the model to a general multi--key modulation scheme:
an $n$--qubit ciphertext obtained by composing several key-controlled
unitaries.
We prove that, per channel use, an $n$--qubit NVADE state can carry at
most $n$ bits of accessible classical information, regardless of the
number or complexity of modulation keys, and we place the existing
single-key NVADE phase encryption and the proposed multi--key scheme
inside this unified framework.
\end{abstract}

\section{NVADE--based communication model}

\subsection{NVADE state preparation}

Let $t \in \mathbb{C}^L$ denote a finite complex sequence obtained from
a mathematical object (e.g.\ truncated series coefficients).
The NVADE map produces a normalized vector
$\bm{\psi} \in \mathbb{C}^N$ as follows:
\begin{enumerate}
  \item Choose an embedding dimension $N = 2^n \ge L$.
  \item Form $w \in \mathbb{C}^N$ by padding or truncating $t$ to length $N$.
  \item Normalize:
  \[
    \bm{\psi} = \frac{w}{\|w\|_2}, \qquad
    \|w\|_2 = \Big( \sum_{k=0}^{N-1} |w_k|^2 \Big)^{1/2}.
  \]
\end{enumerate}
The pair $(t, N)$, together with a fixed ordering convention, uniquely
determines $\bm{\psi}$ up to a global phase.

On hardware, $\bm{\psi}$ is prepared as a pure $n$--qubit state
\[
  |\psi\rangle = \sum_{k=0}^{N-1} \psi_k\, |k\rangle
  \in (\mathbb{C}^2)^{\otimes n},
  \quad N = 2^n.
\]
We denote the corresponding density operator by
$\rho = |\psi\rangle\!\langle \psi|$.

\subsection{NVADE--SEC channel abstraction}

A generic NVADE--SEC channel can be described as follows.
Let $M$ be a classical message random variable taking values
$m \in \{0,1\}^m$, and let $K$ be a (possibly structured) classical key
taking values in a key space $\mathcal{K}$.
For each $(m, K)$ the protocol specifies:
\begin{enumerate}
  \item A NVADE state $|\psi_0(m,K)\rangle$ prepared on $n$ qubits.
  \item A (possibly key-dependent) quantum channel
        $\mathcal{E}_K : \mathcal{D}((\mathbb{C}^2)^{\otimes n})
        \to \mathcal{D}((\mathbb{C}^2)^{\otimes n})$.
\end{enumerate}
The transmitted ciphertext state is
\[
  \rho_{m,K}
  = \mathcal{E}_K\!\big(|\psi_0(m,K)\rangle\!\langle \psi_0(m,K)|\big)
  \in \mathcal{D}((\mathbb{C}^2)^{\otimes n}),
\]
where $\mathcal{D}(\mathcal{H})$ denotes the set of density operators
on the Hilbert space $\mathcal{H}$.

A receiver with full key knowledge has access to a decoding map
$\mathcal{D}_K$ and a measurement $\{M_x\}$ on $n$ qubits, producing
classical outcomes $X$ from which an estimate $\hat{M}$
of the message is recovered.

\section{Information limit for NVADE--SEC channels}

We now make precise the information-theoretic limitation that an $n$--qubit
NVADE--SEC channel cannot deliver more than $n$ bits of classical
information per use, regardless of the structure of the key or the
complexity of the modulation.

\subsection{Ensemble and accessible information}

Fix a joint distribution $p(m,k)$ over $(M,K)$.
The \emph{average} ciphertext state is
\[
  \bar{\rho} = \sum_{m,k} p(m,k) \,\rho_{m,k}.
\]
Conditioned on a particular key value $K=k$, the message ensemble is
$\{p(m|k), \rho_{m,k}\}_m$.

Let $\{M_x\}$ be a POVM on $(\mathbb{C}^2)^{\otimes n}$.
Given outcome $X=x$, the observer obtains information about $M$.
The maximal mutual information between $M$ and $X$ achievable by varying
$\{M_x\}$ is the \emph{accessible information} $I_{\text{acc}}(M:X)$.

By the Holevo bound,
\[
  I_{\text{acc}}(M:X) \le \chi(M)
  := S(\bar{\rho}) - \sum_m p(m) S(\rho_m),
\]
where $S(\rho) = -\mathrm{Tr}(\rho \log_2 \rho)$ is the von Neumann entropy,
and $\rho_m = \sum_k p(k|m) \rho_{m,k}$ is the ciphertext state averaged
over keys conditioned on $M=m$.

\subsection{Information limit per use}

We now specialize to the $n$--qubit setting.

\begin{theorem}[NVADE information limit]
\label{thm:info-limit}
Let $\mathcal{H} = (\mathbb{C}^2)^{\otimes n}$.
Consider any NVADE--SEC channel that, per use, produces ciphertext
states $\rho_m \in \mathcal{D}(\mathcal{H})$ from messages
$m \in \{0,1\}^m$ (possibly via key-dependent maps as above).
Then for any choice of message distribution $p(m)$
and any measurement $\{M_x\}$ on $\mathcal{H}$, the accessible
information satisfies
\[
  I_{\text{acc}}(M:X) \le n.
\]
In particular, the classical capacity per use of the channel is at most
$n$ bits.
\end{theorem}

\begin{proof}
For fixed $p(m)$, the Holevo information
$\chi(M) = S(\bar{\rho}) - \sum_m p(m) S(\rho_m)$ satisfies
\[
  \chi(M) \le S(\bar{\rho}) \le \log_2 \dim \mathcal{H}
  = \log_2 (2^n) = n.
\]
The first inequality uses $S(\rho_m)\ge 0$ for all $m$; the second uses
the standard upper bound $S(\rho) \le \log_2 d$ for any state on a
$d$--dimensional Hilbert space.
By the Holevo bound $I_{\text{acc}}(M:X) \le \chi(M)$, so
$I_{\text{acc}}(M:X) \le n$ for any measurement $\{M_x\}$.

Taking the supremum over all message distributions $p(m)$ and all
measurements yields that the classical capacity per channel use is
upper-bounded by $n$ bits.
\end{proof}

\begin{corollary}[Keys do not increase per-use capacity]
\label{cor:keys-no-extra-bits}
In the setting of Theorem~\ref{thm:info-limit}, allow arbitrary
classical keys $K$ and arbitrary key-dependent channels
$\mathcal{E}_K$ and decoders $\mathcal{D}_K$.
Even if $K$ is chosen from an arbitrarily large key space and jointly
distributed with $M$, the mutual information between $M$ and any
measurement outcome on the $n$--qubit ciphertext, conditioned on $K$,
is still bounded by $n$:
\[
  I_{\text{acc}}(M:X \mid K) \le n.
\]
Thus additional keys cannot increase the per-use payload beyond $n$ bits;
they affect only security, not capacity.
\end{corollary}

\begin{proof}
Fix $K=k$.
Conditioned on $K=k$, we obtain an ensemble
$\{p(m|k), \rho_{m,k}\}_m$ of states on $\mathcal{H}$.
Applying Theorem~\ref{thm:info-limit} to this ensemble yields
$I_{\text{acc}}(M:X \mid K=k) \le n$.
Taking the expectation over $K$ preserves the upper bound.
\end{proof}

\section{Multi--key modulation channels}

We now introduce the multi--key structure used by NVADE--SEC for
encryption and key splitting.

\subsection{Key modulators}

Let $\mathcal{H} = (\mathbb{C}^2)^{\otimes n}$ as before.
Let $K_1,\dots,K_k$ be classical keys, with values in key spaces
$\mathcal{K}_1,\dots,\mathcal{K}_k$.
For each $i$ and each $K_i \in \mathcal{K}_i$, let
\[
  M_i(K_i) : \mathcal{H} \to \mathcal{H}
\]
be a unitary operator (typically diagonal or structured in the NVADE
basis).
We call $M_i(K_i)$ a \emph{key modulator}.

Given a base preparation $|\psi_0(m)\rangle$ for message $m$, the
multi--key modulation channel applies the composite unitary
\[
  U_K := M_k(K_k) \cdots M_2(K_2) M_1(K_1),
  \qquad K = (K_1,\dots,K_k),
\]
to produce the ciphertext
\[
  |\psi_{\text{ct}}(m,K)\rangle = U_K |\psi_0(m)\rangle,
  \qquad
  \rho_{m,K} = |\psi_{\text{ct}}(m,K)\rangle\!\langle \psi_{\text{ct}}(m,K)|.
\]

A legitimate receiver with access to all keys $K_1,\dots,K_k$ can
invert $U_K$ by applying $U_K^\dagger$ followed by a decoding
measurement tailored to $|\psi_0(m)\rangle$.

\subsection{Information limit in the multi--key setting}

The multi--key modulation channel is a special case of the general
NVADE--SEC channel described above, with
\[
  \mathcal{E}_K(\rho) = U_K\, \rho\, U_K^\dagger.
\]
Thus Theorem~\ref{thm:info-limit} and
Corollary~\ref{cor:keys-no-extra-bits} apply directly.

\begin{proposition}[Multi--key information limit]
\label{prop:multikey-limit}
Consider the multi--key modulation channel defined above on $n$ qubits.
For any number of keys $k \ge 1$, any choice of message distribution
$p(m)$, key distribution $p(K)$, and any measurement $\{M_x\}$ on
$\mathcal{H}$, the accessible information per use satisfies
\[
  I_{\text{acc}}(M:X \mid K) \le n.
\]
\end{proposition}

\begin{proof}
Immediate from Corollary~\ref{cor:keys-no-extra-bits}, observing that
$\mathcal{E}_K$ is unitary and acts on $\mathcal{H}=(\mathbb{C}^2)^{\otimes n}$.
\end{proof}

In particular, multi--key modulation cannot increase the per-use payload
beyond $n$ bits; it can only redistribute how those bits are encoded
(e.g.\ among multiple logical fields) and how hard they are to recover
without the full key set.

\section{Security notions for missing-key adversaries}

The role of additional keys is to degrade an adversary's ability to
distinguish messages when some subset of keys is unknown.

\subsection{IND--style experiment}

We sketch a standard indistinguishability game tailored to the
multi--key setting.

\medskip
\noindent
\textbf{Game: IND--multikey.}
\begin{enumerate}
  \item The adversary $\mathcal{A}$ outputs two messages
        $(m_0,m_1) \in \{0,1\}^m\times\{0,1\}^m$.
  \item The challenger samples a key tuple
        $K = (K_1,\dots,K_k) \sim P_K$ and a secret bit $b\in\{0,1\}$.
  \item The challenger prepares the ciphertext state
        $\rho_{m_b,K}$ and returns it to $\mathcal{A}$.
        Optionally, $\mathcal{A}$ may be given some subset $S\subset\{1,\dots,k\}$
        of the key components (e.g.\ $K_i$ for $i\in S$) and promised
        that at least one key is missing ($S \neq \{1,\dots,k\}$).
  \item The adversary performs a measurement on the received state
        (and any known keys) and outputs a guess $b'$.
\end{enumerate}
The \emph{advantage} of $\mathcal{A}$ is
\[
  \mathrm{Adv}_{\text{IND}}(\mathcal{A})
  = \big| \Pr[b' = b] - \tfrac{1}{2} \big|.
\]

We say the multi--key scheme is \emph{IND--secure against missing-key
adversaries} if, for all efficient $\mathcal{A}$ having access to any
proper subset of the keys, $\mathrm{Adv}_{\text{IND}}(\mathcal{A})$ is
negligible in a chosen security parameter.

\subsection{Design goal}

From the perspective of NVADE--SEC, the design goal is:

\medskip
\noindent
\textbf{Goal.}
Choose the families of modulators $M_i(K_i)$ such that:
\begin{enumerate}
  \item For a legitimate receiver with all keys
        $K_1,\dots,K_k$, the decoding map recovers up to $n$ bits of
        message with high fidelity per use; and
  \item For any adversary missing at least one key, the ensemble of
        ciphertext states $\{\rho_{m,K} : m\in\{0,1\}^m\}$ is
        (information-theoretically or computationally) close to a
        key-independent reference ensemble (e.g.\ a maximally mixed or
        high-entropy distribution), so that
        $\mathrm{Adv}_{\text{IND}}(\mathcal{A})$ is provably small.
\end{enumerate}

In this formulation, additional keys do not change the upper bound
$n$ on the per-use payload; they increase the difficulty of the
distinguishing problem for adversaries lacking some portion of the key
tuple $K$.

\section{Placement in the NVADE--SEC spine}

The results above provide the theoretical backbone for the encryption
units in the NVADE--SEC spine.  In particular:
\begin{itemize}
  \item \textbf{E0 (Key \& channel spec).} Formalizes the NVADE
        key bundle and channel description, including the mapping
        from classical series metadata to $n$--qubit NVADE states.
  \item \textbf{E1 (Information limit).}
        Theorem~\ref{thm:info-limit} and
        Proposition~\ref{prop:multikey-limit} establish that any
        NVADE--SEC channel on $n$ qubits has per-use classical
        capacity at most $n$ bits, even in the presence of arbitrarily
        many key modulators.
  \item \textbf{E1--T1 (Classical XOR baseline).}
        A purely classical one-time-pad style scheme providing a
        comparison point for NVADE-based encryption.
  \item \textbf{E1--T2 (Single-key NVADE phase encryption).}
        A concrete instantiation with one diagonal NVADE-derived
        phase key $M_1(K_1)$, empirically validated by hardware and
        simulator runs.
  \item \textbf{E1--T3 (Multi-key NVADE modulation, planned).}
        A forthcoming unit that instantiates the above multi--key
        model with $k\ge 2$ independent key modulators and explores
        the tradeoff between reliability (decoding fidelity) and
        security (advantage of missing-key adversaries) under the
        global $n$--bit information limit.
\end{itemize}

Together, these results guarantee that the encryption layer of
NVADE--SEC is grounded in a clean, dimension-aware information
theory: $n$ qubits provide at most $n$ bits of accessible classical
information per use, and every additional key must be justified as a
security feature, not as a way to exceed that fundamental limit.

\end{document}

