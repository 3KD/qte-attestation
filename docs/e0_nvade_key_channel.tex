\documentclass[11pt]{article}
\usepackage{amsmath,amssymb,amsthm}

\title{E0 -- NVADE Key, Channel, and Threat Model}
\author{QTE / NVADE--SEC}
\date{}

\begin{document}
\maketitle

\section*{Hilbert space and encoder family}

For $n \in \mathbb{N}$, let $\mathcal{H}_n$ be the $n$--qubit Hilbert space
\[
  \mathcal{H}_n \coloneqq \mathrm{span}\{ \lvert x\rangle : x \in \{0,1\}^n \},
\]
with computational basis $\{ \lvert x\rangle \}_{x \in \{0,1\}^n}$.

A \emph{NVADE encoder family} of $n$--qubit states is a map
\[
  F : \Theta \longrightarrow \mathbb{C}^{2^n}, \qquad
  \theta \longmapsto \psi(\theta),
\]
together with metadata describing:
\begin{itemize}
  \item a series identifier $\mathrm{series\_id}$ (analytic family),
  \item a truncation rule determining $n$ and the number of series coefficients,
  \item a weighting scheme on coefficients,
  \item an index order mapping term indices to basis elements $\lvert x\rangle$,
  \item a normalization convention.
\end{itemize}
The normalized statevector associated to $\theta \in \Theta$ is
\[
  \lvert \psi(\theta)\rangle \coloneqq
  \frac{\psi(\theta)}{\|\psi(\theta)\|_2} \in \mathcal{H}_n.
\]

\section*{Measurement distribution}

Given $\theta \in \Theta$, the (ideal) measurement distribution in the computational basis is
\[
  p_\theta(x) \coloneqq \bigl|\langle x \mid \psi(\theta)\rangle\bigr|^2,
  \qquad x \in \{0,1\}^n,
\]
which satisfies $\sum_{x} p_\theta(x) = 1$.

\section*{Public family and secret key}

\paragraph{Public NVADE family.}
A \emph{public NVADE family} is the data
\[
  \mathsf{Family} \coloneqq
  \bigl(
    \mathrm{family\_id},\;
    \mathrm{series\_id},\;
    n,\;
    \mathrm{max\_terms},\;
    \mathrm{weighting\_scheme},\;
    \mathrm{index\_order},\;
    \mathrm{amp\_norm},\;
    \mathrm{extra\_metadata}
  \bigr),
\]
fixing a particular encoder family $F : \Theta \to \mathbb{C}^{2^n}$ and the basis ordering
$x \mapsto \lvert x\rangle$.

\paragraph{Secret key.}
A \emph{secret NVADE key} is a choice of parameter
\[
  \theta \in \Theta
\]
for the fixed family $F$. In practice we treat $\theta$ as an opaque payload
(e.g.\ hidden series parameters, seeds, branch choices) such that distinct keys yield
typically distinct normalized states $\lvert \psi(\theta)\rangle$ up to global phase.

We denote a secret key by the pair
\[
  \mathsf{Key} \coloneqq (\mathsf{Family}, \theta).
\]

\section*{Honest channel specification}

The honest implementation of a NVADE key on a backend $B$ aims to realize samples from
$p_\theta$ (possibly distorted by noise).

\paragraph{Backends.}
We distinguish two backend kinds:
\begin{itemize}
  \item $\mathsf{AER}$: a simulator backend (e.g.\ Qiskit Aer),
  \item $\mathsf{IBM}$: an IBM Cloud hardware or noisy-simulator backend
        (e.g.\ a Qiskit Runtime backend such as \texttt{ibm\_torino}).
\end{itemize}

\paragraph{Channel spec.}
An \emph{honest channel specification} for a fixed public family
$\mathsf{Family}$ is a tuple
\[
  \mathsf{Chan} \coloneqq
  \bigl(
    \mathsf{Family},\;
    \mathsf{backend\_kind} \in \{\mathsf{AER},\mathsf{IBM}\},\;
    \mathsf{backend\_name},\;
    n,\;
    \mathsf{shots},\;
    \mathsf{measurement\_basis},\;
    \mathsf{noise\_tag},\;
    \mathsf{extra\_config}
  \bigr),
\]
subject to the consistency condition
\[
  n = \mathrm{truncation\_n}(\mathsf{Family}).
\]

Mathematically, the honest channel corresponds to a (possibly noisy) CPTP map
\[
  \Lambda : \mathcal{D}(\mathcal{H}_n) \longrightarrow \mathcal{D}(\mathcal{H}_n),
\]
where $\mathcal{D}(\mathcal{H}_n)$ denotes density operators on $\mathcal{H}_n$.
Ideally, measurement outcomes are drawn from
\[
  p_\theta^\Lambda(x)
  \coloneqq \operatorname{Tr}\!\bigl[\Lambda(\lvert \psi(\theta)\rangle\langle\psi(\theta)\rvert)
                                   \lvert x\rangle\langle x\rvert\bigr],
  \qquad x \in \{0,1\}^n.
\]

\section*{Adversary access and threat model E0}

We describe the adversary at E0 purely in terms of their \emph{view}, without
committing to a specific security game.

\paragraph{Adversary access modes.}
For a fixed public family $\mathsf{Family}$ and channel spec $\mathsf{Chan}$, we consider
the following access modes:

\begin{itemize}
  \item $\mathsf{PUBLIC\_SPEC\_ONLY}$: adversary sees only $\mathsf{Family}$.
  \item $\mathsf{PUBLIC\_SPEC\_AND\_SAMPLES}$: adversary sees $\mathsf{Family}$ and can
        obtain (bounded) samples from $p_\theta^\Lambda$ by querying the channel.
  \item $\mathsf{PUBLIC\_SPEC\_AND\_NOISY\_MODEL}$: adversary knows an explicit approximate
        noise model $\Lambda_{\mathrm{approx}}$ and can simulate approximate
        $p_\theta^{\Lambda_{\mathrm{approx}}}$.
  \item $\mathsf{FULL\_CHANNEL\_TRANSCRIPT}$: adversary sees everything that a verifier
        sees in an attestation protocol (full measurement transcripts, meta-data),
        but does not learn $\theta$.
\end{itemize}

\paragraph{Threat model E0.}
An \emph{E0 threat model} for NVADE--SEC is a tuple
\[
  \mathsf{Threat} \coloneqq
  \bigl(
    \mathsf{Family},\;
    \mathsf{Chan},\;
    \mathsf{Access},\;
    \mathsf{notes}
  \bigr),
\]
where $\mathsf{Access}$ is one of the modes above and \emph{notes} is a natural-language
record of additional assumptions (e.g.\ bounds on the number of samples).

\section*{NVADE key bundle}

For downstream units (U31, U56, U58) we package the objects above into a single
\emph{NVADE key bundle}
\[
  \mathsf{Bundle} \coloneqq
  \bigl(
    \mathsf{Family},\;
    \mathsf{Key},\;
    \mathsf{Chan},\;
    \mathsf{Threat}
  \bigr),
\]
where all components share the same public family and satisfy the consistency
conditions described above.

The E0 specification is purely classical: it fixes the mathematical object
$\lvert \psi(\theta)\rangle$, the intended implementation $\Lambda$, and the
adversary's observational capabilities. Concrete security experiments (e.g.\ trapdoor
witnesses, noise-window advantage, attestation cards) are defined on top of this bundle
in subsequent units.
\end{document}
