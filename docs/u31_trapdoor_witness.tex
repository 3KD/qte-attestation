\documentclass[11pt]{article}

\usepackage[margin=1in]{geometry}
\usepackage{amsmath,amssymb,amsthm}
\usepackage{bm}
\usepackage{hyperref}

\newtheorem{definition}{Definition}
\newtheorem{proposition}{Proposition}

\title{U31--T1 Trapdoor Witness ROC for NVADE--SEC Phase Encryption}
\author{E.~Dollarhide}
\date{\today}

\begin{document}
\maketitle

\begin{abstract}
Unit U31--T1 implements a concrete ``trapdoor witness'' test for NVADE--SEC
phase encryption.
Given a reference amplitude state $\bm{\psi} \in \mathbb{C}^{2^n}$ and an
impostor state $\bm{\psi}^{(\mathrm{imp})}$ on the same $n$ qubits, we define
a log--likelihood score for observed bitstrings with respect to the
reference distribution $p(k) = |\psi_k|^2$ and measure how well these scores
separate honest vs impostor experiments.
The test runs both on Qiskit Aer and on an IBM Quantum backend
(\texttt{ibm\_fez}), and reports ROC metrics
(AUC, TPR at $1\%$ FPR) together with summary statistics.
This unit is the first concrete implementation of the U31 trapdoor witness
in the NVADE--SEC spine.
\end{abstract}

\section{Setup: reference and impostor states}

Let $n \ge 1$ and $N = 2^n$.
We fix a reference amplitude vector
\[
  \bm{\psi} = (\psi_0,\dots,\psi_{N-1}) \in \mathbb{C}^N,
  \qquad
  \|\bm{\psi}\|_2 = 1,
\]
prepared from a NVADE series embedding and stored on disk as
\verb|states/u31_ref_psi.npy|.
This defines an $n$--qubit pure state
\[
  |\psi\rangle = \sum_{k=0}^{N-1} \psi_k \, |k\rangle,
  \qquad
  p(k) = |\psi_k|^2.
\]

An impostor state $\bm{\psi}^{(\mathrm{imp})} \in \mathbb{C}^N$ is either
\begin{itemize}
  \item provided explicitly via a second \verb|.npy| file, or
  \item generated synthetically by applying a random permutation and
        random phases to $\bm{\psi}$:
  \[
    \psi^{(\mathrm{imp})}_k = \psi_{\pi(k)} e^{i\theta_k}, \quad
    \theta_k \sim \mathrm{Unif}[0,2\pi).
  \]
\end{itemize}
In U31--T1 we use the second option, so that the impostor ensemble
approximates a ``wrong key'' state family derived from the same spectrum.

\section{Score function and ROC statistic}

\subsection{Log--likelihood scores}

\begin{definition}[Log--likelihood score]
Given a reference distribution $p(k) = |\psi_k|^2$ over
$\{0,1\}^n$ and an observed bitstring $x \in \{0,1\}^n$,
the U31--T1 score is
\[
  s(x) = \log\big(p(x) + \varepsilon\big),
\]
where $\varepsilon > 0$ is a small regulator (fixed to $\varepsilon = 10^{-15}$
in the implementation) to avoid $\log 0$.
\end{definition}

For an experiment producing a multiset of bitstrings with empirical
counts $\{c_x\}$, we expand this into a score sample
\[
  \mathcal{S}
  = \{ s(x) : x \text{ observed, repeated } c_x \text{ times}\}.
\]
In U31--T1 we obtain:
\begin{itemize}
  \item an ``honest'' score set $\mathcal{S}_+$ from runs prepared with
        the reference state $\bm{\psi}$, and
  \item an ``impostor'' score set $\mathcal{S}_-$ from runs prepared with
        $\bm{\psi}^{(\mathrm{imp})}$.
\end{itemize}

\subsection{ROC curve and AUC}

Let $\{s_i^+\} \subset \mathcal{S}_+$ and $\{s_j^-\} \subset \mathcal{S}_-$
denote the honest and impostor scores.
We define labels $y = 1$ for honest and $y = 0$ for impostor, and sort the
pooled scores $\{(s_\ell, y_\ell)\}$ in descending order of $s_\ell$.

For any threshold $\tau \in \mathbb{R}$, the corresponding classifier
\[
  \hat{y}_\tau = \mathbf{1}\{ s \ge \tau \}
\]
yields a true positive rate and false positive rate
\[
  \mathrm{TPR}(\tau) = \frac{\#\{s_i^+ \ge \tau\}}{\#\mathcal{S}_+},
  \qquad
  \mathrm{FPR}(\tau) = \frac{\#\{s_j^- \ge \tau\}}{\#\mathcal{S}_-}.
\]
Sweeping $\tau$ from $+\infty$ to $-\infty$ traces out a ROC curve
$(\mathrm{FPR}(\tau), \mathrm{TPR}(\tau))$.
The area under this curve is estimated numerically by trapezoidal rule:
\[
  \mathrm{AUC}
  \approx \sum_{i=1}^{L-1}
  \big( \mathrm{FPR}_i - \mathrm{FPR}_{i-1} \big)
  \frac{\mathrm{TPR}_i + \mathrm{TPR}_{i-1}}{2}.
\]

We also report the true positive rate at $1\%$ false positive rate,
\[
  \mathrm{TPR}_{1\%}
  = \max\{ \mathrm{TPR}(\tau) : \mathrm{FPR}(\tau) \le 0.01\},
\]
as a more operational threshold for key-accept / key-reject decisions.

\section{Implementation details}

\subsection{Runner script}

The U31--T1 harness is implemented in
\verb|runner/u31_t1_trapdoor_witness.py|.
Its main arguments are:
\begin{itemize}
  \item \verb|--ref-psi|: path to \verb|.npy| file with the reference
        amplitude vector (complex dtype).
  \item \verb|--alt-psi| (optional): path to impostor vector; if omitted,
        a permuted / rephased version of \verb|ref-psi| is generated.
  \item \verb|--backend|: \verb|aer| or \verb|ibm|.
  \item \verb|--backend-name|: optional explicit backend name
        (e.g.\ \verb|ibm_fez|).
  \item \verb|--shots|: shots per class (honest and impostor).
  \item \verb|--seed|: base RNG seed.
  \item \verb|--send-ibm|: required guardrail flag when using
        \verb|--backend ibm|.
  \item \verb|--out|: output JSON path.
\end{itemize}

The script performs the following steps:
\begin{enumerate}
  \item Load \verb|ref-psi| (and \verb|alt-psi| if provided); otherwise,
        construct $\bm{\psi}^{(\mathrm{imp})}$ by a random permutation
        and random phases.
  \item Build a probability map $p(k)$ from \verb|ref-psi|.
  \item Construct an $n$--qubit circuit using \verb|QuantumCircuit.initialize|
        to prepare the given amplitude vector, then measure all qubits
        in the computational basis.
  \item Choose a backend:
        \begin{itemize}
          \item \textbf{Aer}: \verb|qiskit-aer| simulator
                (\verb|aer_simulator|).
          \item \textbf{IBM}: a real device obtained from
                \verb|QiskitRuntimeService()|, here pinned to
                \verb|ibm_fez|.
        \end{itemize}
  \item Run honest and impostor circuits with the same number of shots,
        obtaining counts dictionaries \verb|counts_ref| and
        \verb|counts_alt|.
  \item Convert counts into score samples using $\log p(k)$, then compute
        ROC, AUC, and $\mathrm{TPR}_{1\%}$.
  \item Save a JSON payload containing:
        \begin{itemize}
          \item backend kind and name,
          \item shot count and seed,
          \item ROC arrays and area,
          \item $\mathrm{TPR}_{1\%}$,
          \item summary statistics (mean and standard deviation) of honest
                and impostor scores.
        \end{itemize}
\end{enumerate}

The script also prints a human-readable summary to stdout, which is
included below for the canonical runs.

\subsection{Canonical Aer run}

The canonical simulator run uses
\begin{itemize}
  \item \verb|backend = aer_simulator|,
  \item \verb|n_qubits = 3|,
  \item \verb|shots_per_class = 4096|,
  \item \verb|seed = 424242|.
\end{itemize}
The command is:
\begin{verbatim}
python runner/u31_t1_trapdoor_witness.py \
  --ref-psi states/u31_ref_psi.npy \
  --backend aer \
  --shots 4096 \
  --seed 424242 \
  --out runs/u31_t1_aer.json
\end{verbatim}

The observed summary was:
\begin{verbatim}
== U31-T1 trapdoor witness ==
 file: runs/u31_t1_aer.json
 backend_kind: aer
 backend_name: aer_simulator
 shots_per_class: 4096
 n_qubits: 3
 auc: 0.6783105731010437
 tpr_at_1pct_fpr: 0.193603515625
 scores_summary:
  honest: count=4096, mean=-1.813526, std=0.679718
  impostor: count=4096, mean=-2.362832, std=0.546601
\end{verbatim}

\subsection{Canonical IBM hardware run}

The canonical hardware run uses
\begin{itemize}
  \item backend \verb|ibm_fez|,
  \item \verb|n_qubits = 3|,
  \item \verb|shots_per_class = 4096|,
  \item \verb|seed = 424242|,
  \item explicit opt-in via \verb|--send-ibm|.
\end{itemize}
The command is:
\begin{verbatim}
python runner/u31_t1_trapdoor_witness.py \
  --ref-psi states/u31_ref_psi.npy \
  --backend ibm \
  --backend-name ibm_fez \
  --shots 4096 \
  --seed 424242 \
  --send-ibm \
  --out runs/u31_t1_ibm_keyed.json
\end{verbatim}

The observed summary was:
\begin{verbatim}
== U31-T1 trapdoor witness ==
 file: runs/u31_t1_ibm_keyed.json
 backend_kind: ibm
 backend_name: ibm_fez
 shots_per_class: 4096
 n_qubits: 3
 auc: 0.6823735237121582
 tpr_at_1pct_fpr: 0.215087890625
 scores_summary:
  honest: count=4096, mean=-1.812453, std=0.670984
  impostor: count=4096, mean=-2.345398, std=0.574837
\end{verbatim}

Both runs show an AUC significantly above $0.5$ (random guessing) and a
nontrivial $\mathrm{TPR}_{1\%}$, demonstrating that the log--likelihood
scores derived from the NVADE amplitude distribution $p(k)$ contain a
usable trapdoor signal: honest data are statistically distinguishable
from impostor data with a simple scalar test.

\section{Placement in the NVADE--SEC spine}

Unit U31--T1 sits in the NVADE--SEC spine as follows:
\begin{itemize}
  \item Upstream, R1 provides a finite-sample entropy/min-entropy
        certificate for classical bitstreams, and U58 specifies the
        overall attestation card.
  \item E0/E1 formalize the NVADE--SEC key/channel model and
        information limit (at most $n$ bits per $n$-qubit use).
  \item U31--T1 instantiates a concrete trapdoor witness:
        given a reference amplitude state and a candidate impostor
        state on the same qubits, a single experiment yields ROC
        metrics quantifying how strongly the data support the
        ``correct key'' hypothesis.
  \item Future unit U31--T2 will replace the synthetic impostor with
        a structured wrong-key family and log-likelihood ratio tests
        over that key ensemble, but U31--T1 already provides a
        working, hardware-validated witness for the trapdoor signal.
\end{itemize}

\end{document}
