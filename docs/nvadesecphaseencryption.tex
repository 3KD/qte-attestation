\documentclass[11pt]{article}
\usepackage{amsmath,amssymb}
\usepackage[T1]{fontenc}
\usepackage[utf8]{inputenc}

\title{A NVADE--SEC Phase-Mask Primitive with $n$-Bit Amplitude Information Limit}
\author{Erik Dollarhide}
\date{\today}

\begin{document}
\maketitle

\begin{abstract}
We define a simple Hilbert-space phase-mask primitive built on Normalized Vector Amplitude Distribution Encoding (NVADE). A NVADE key is any $n$-qubit pure state obtained by applying NVADE to a convergent complex series or other structured object. We specify a family of message-dependent diagonal unitaries that modulate the phases of the key, derive an $n$-bit amplitude information limit for this construction (at most $n$ bits of classical message per $n$-qubit NVADE key), and report an initial correctness experiment showing unit-fidelity decryption for a six-qubit, six-bit instance implemented in the \texttt{qte-attestation} codebase. We explicitly treat this as a phase-mask \emph{primitive}---not a full IND-CPA-secure encryption scheme---and leave full security analysis to future work.
\end{abstract}

\section{NVADE keys}

Let $f(k) \in \mathbb{C}$ be a convergent complex series or other discrete structure indexed by $k \in \mathbb{N}$. NVADE fixes
\begin{itemize}
  \item an ordering of indices $k = 0,\dots,2^n-1$,
  \item a weighting convention $w_k$,
  \item and a normalization map from the complex vector to an $n$-qubit statevector.
\end{itemize}
The resulting amplitudes
\[
  \psi_x
  = \frac{w_x f(x)}{\sqrt{\sum_{y} |w_y f(y)|^2}},
  \qquad x \in \{0,1\}^n
\]
define an $n$-qubit pure state
\[
  \lvert \psi_{\mathrm{key}} \rangle
  = \sum_{x \in \{0,1\}^n} \psi_x \lvert x \rangle,
\]
which we call a \emph{NVADE key state}. In the implementation used for the experiment below we take $f$ to be a truncated series for $\pi$, but the construction is agnostic to the choice of $f$: any NVADE embedding that produces a normalized statevector can serve as the key.

\section{Phase-mask primitive}

Fix $n$ qubits and identify the computational basis with $\{0,1\}^n$. For any message bitstring $m \in \{0,1\}^m$ with $m \le n$, define a diagonal unitary
\[
  U_m = \mathrm{diag}\big( (-1)^{\langle x, m \rangle} \big)_{x \in \{0,1\}^n},
\]
where $\langle x, m \rangle$ is the bitwise inner product modulo two. Given a NVADE key state $\lvert \psi_{\mathrm{key}} \rangle$, the associated \emph{cipherstate} for message $m$ is
\[
  \lvert \psi_{\mathrm{enc}}(m) \rangle
  = U_m \lvert \psi_{\mathrm{key}} \rangle.
\]
Applying the same unitary again yields
\[
  \lvert \psi_{\mathrm{dec}}(m) \rangle
  = U_m \lvert \psi_{\mathrm{enc}}(m) \rangle
  = U_m^2 \lvert \psi_{\mathrm{key}} \rangle
  = \lvert \psi_{\mathrm{key}} \rangle,
\]
since each phase factor squares to $+1$. In circuit form, $U_m$ can be realized as a product of $Z$-type phase gates controlled by the message bits.

This construction is what we refer to as the NVADE--SEC phase-mask primitive (E1--T2): keys are structured NVADE embeddings, while messages act as phase masks on those amplitude keys. In this note we focus on \emph{correctness} and \emph{capacity} of this primitive, and do not claim full semantic security.

\section{Amplitude information limit}

For this fixed family of phase masks, the usable message space per application is at most $n$ bits. On $n$ qubits each mask is determined by some $m \in \{0,1\}^n$, and the map
\[
  m \longmapsto U_m
\]
is linear over $\mathbb{F}_2^n$. The set of distinct unitaries $\{U_m\}$ therefore has cardinality $2^n$, and the corresponding message capacity is
\[
  \log_2 \lvert \{U_m\} \rvert = n \quad \text{bits per use.}
\]
In other words, although the underlying Hilbert space has exponential dimension $2^n$, this particular NVADE--SEC primitive explores an $n$-bit family of diagonal unitaries acting on a fixed NVADE key state. We refer to this $n$-bit bound as the \emph{amplitude information limit} of the scheme.

We emphasize that this is a capacity statement for the specific family of diagonal masks $\{U_m\}$ used here, not for arbitrary unitaries on the $n$-qubit Hilbert space.

\section{Threat model and limitations}

We treat the construction above as a \emph{single-use phase-mask primitive} rather than a complete, multi-use encryption scheme.

\begin{itemize}
  \item \textbf{Correctness focus.} In this note we prove and test only \emph{correctness}: for any fixed NVADE key and message $m$, applying $U_m$ twice recovers the original key state with unit fidelity in the ideal (noiseless) model.
  \item \textbf{Amplitude profile not secret.} The amplitude profile $\{|\psi_x|^2\}$ of the NVADE key is invariant under $U_m$ and may be treated as public; messages are encoded solely in phases. Any computational-basis measurement of a single cipherstate is independent of $m$.
  \item \textbf{Single-use per key state.} Because different messages correspond to different diagonal masks acting on the \emph{same} key state, multiple uses with different masks may leak structured information about mask differences (e.g.\ via interference or swap-type tests). In this preliminary note we therefore restrict attention to a \emph{single logical use} of the primitive per prepared key state and leave quantitative analysis of key reuse to future work.
  \item \textbf{No full security definition.} We do not formalize an IND-CPA-style adversarial game or prove semantic security; doing so would require a more detailed treatment of the adversary's access to copies of the key state and to multiple ciphertexts. Here we use the word ``primitive'' rather than ``encryption scheme'' to reflect this limitation.
  \item \textbf{No noise analysis.} All experiments reported below are performed in the ideal statevector model. Hardware and noise robustness studies are left to separate work within the broader NVADE--SEC programme.
\end{itemize}

\section{Implementation and experiment}

We implemented this primitive in the \texttt{qte-attestation} repository as test E1--T2. A NVADE key state with $n = 6$ qubits is constructed from a truncated series for $\pi$ and stored as a normalized statevector. For each trial we:
\begin{enumerate}
  \item sample a random message $m \in \{0,1\}^6$,
  \item apply $U_m$ to obtain $\lvert \psi_{\mathrm{enc}}(m) \rangle$,
  \item apply $U_m$ again to obtain $\lvert \psi_{\mathrm{dec}}(m) \rangle$,
  \item compute the fidelity
    \[
      F(m) = \left\lvert
      \langle \psi_{\mathrm{key}} \mid \psi_{\mathrm{dec}}(m) \rangle
      \right\rvert^2.
    \]
\end{enumerate}

In a reference run with
\[
  n = 6,\quad m = 6,\quad N = 5000 \text{ trials},
\]
the summary statistics recorded in
\texttt{runs/e1\_t2\_n6\_m6\_n5000.json} are
\begin{align*}
  \texttt{average\_fidelity} &\approx 1.0000000000000002,\\
  \texttt{min\_fidelity} &\approx 1.0000000000000004,
\end{align*}
up to floating-point rounding. In all sampled trials the recovered state is numerically identical to the original NVADE key state, confirming correctness of the implemented phase-mask unitaries for all tested messages in the noiseless model.

\section{Relation to prior work and outlook}

Diagonal phase masks and fixed reference states have appeared previously in the context of quantum stream ciphers and data-locking constructions; our goal here is not to introduce diagonal masks per se. The novelty of this note is to show that NVADE key states---structured embeddings of mathematical objects into Hilbert space---support a simple, reproducible phase-mask primitive with a clean $n$-bit amplitude information limit and a concrete implementation in an open-source codebase.

This isolates a minimal NVADE--SEC building block:
\begin{itemize}
  \item keys are structured NVADE embeddings of mathematical objects into Hilbert space;
  \item messages modulate key phases via a family of diagonal unitaries $U_m$;
  \item the amplitude information limit for this primitive is $n$ bits on $n$ qubits;
  \item a reference implementation and reproducible numerical experiment are available in the public codebase.
\end{itemize}
Future work will combine this primitive with explicit adversarial games, error-correcting structure, and handshake protocols, and integrate it into the broader NVADE--SEC attestation and noise-window programme.
\end{document}
