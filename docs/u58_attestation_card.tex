\documentclass[11pt]{article}
\usepackage{amsmath,amssymb,amsthm}

\title{U58 -- NVADE--SEC Attestation Cards}
\author{QTE / NVADE--SEC}
\date{}

\begin{document}
\maketitle

\section*{Preliminaries: NVADE key bundle (E0)}

Fix $n \in \mathbb{N}$ and the $n$--qubit Hilbert space
\[
  \mathcal{H}_n \coloneqq \mathrm{span}\{ \lvert x\rangle : x \in \{0,1\}^n \},
\]
with computational basis $\{ \lvert x\rangle \}_{x \in \{0,1\}^n}$.

A NVADE encoder family is a map
\[
  F : \Theta \longrightarrow \mathbb{C}^{2^n},
  \qquad \theta \longmapsto \psi(\theta),
\]
together with public metadata (family id, series id, truncation rule, weighting
scheme, index order, normalization convention). The normalized statevector is
\[
  \lvert \psi(\theta)\rangle
  \coloneqq \frac{\psi(\theta)}{\|\psi(\theta)\|_2} \in \mathcal{H}_n,
\]
and the associated ideal measurement distribution in the computational basis is
\[
  p_\theta(x) \coloneqq \bigl|\langle x \mid \psi(\theta)\rangle\bigr|^2,
  \qquad x \in \{0,1\}^n.
\]

Following E0, a \emph{NVADE key bundle} is a tuple
\[
  \mathsf{Bundle} \coloneqq
  \bigl(
    \mathsf{Family},\;
    \mathsf{Key},\;
    \mathsf{Chan},\;
    \mathsf{Threat}
  \bigr),
\]
where:
\begin{itemize}
  \item $\mathsf{Family}$ is the public NVADE family description,
  \item $\mathsf{Key} = (\mathsf{Family}, \theta)$ is a secret choice of parameter
        $\theta \in \Theta$ (typically yielding a distinct state up to global phase),
  \item $\mathsf{Chan}$ specifies an honest implementation of $p_\theta$ (or a noisy
        variant $p_\theta^\Lambda$) on a backend of kind $\mathsf{AER}$ or $\mathsf{IBM}$,
  \item $\mathsf{Threat}$ is an E0 threat model specifying what the adversary can observe
        (e.g.\ public spec only, public spec plus samples, full transcript).
\end{itemize}
We write $\mathsf{Bundle}^{\mathrm{pub}}$ for the \emph{public projection} of the bundle,
which exposes only the public family, the channel spec, the adversary access mode and
a key label, but not the internal representation of $\theta$.

\section*{Attestation experiment}

An \emph{attestation experiment} is a single execution of a fixed test
(e.g.\ a trapdoor witness game) using a fixed NVADE key bundle.

At a high level, one attestation experiment consists of:
\begin{itemize}
  \item a choice of unit and test identifier (e.g.\ unit U31, test T2),
  \item a backend configuration (AER or specific IBM device, shot count, noise tag),
  \item a witness construction that computes scores on honest vs.\ impostor data
        and yields a ROC curve with summary metrics,
  \item the resulting measurement data and derived statistics.
\end{itemize}

Formally, we collect the experiment metadata into the following components.

\paragraph{Test descriptor.}
A \emph{test descriptor} is a tuple
\[
  \mathsf{Test} \coloneqq
  \bigl(
    \mathrm{unit\_id},\;
    \mathrm{test\_id},\;
    \mathrm{version},\;
    \mathrm{commit\_hash},\;
    \mathrm{timestamp}
  \bigr),
\]
where:
\begin{itemize}
  \item $\mathrm{unit\_id}$ is a symbolic identifier such as \(\text{``U31''}\),
  \item $\mathrm{test\_id}$ refines the unit (e.g.\ \(\text{``T2-wrong-key-LLR''}\)),
  \item $\mathrm{version}$ records a test/witness version string,
  \item $\mathrm{commit\_hash}$ is an optional VCS commit hash of the code used,
  \item $\mathrm{timestamp}$ is an ISO-8601 timestamp in UTC.
\end{itemize}

\paragraph{Measurement configuration.}
A \emph{measurement configuration} is
\[
  \mathsf{Meas} \coloneqq
  \bigl(
    n,\;
    \mathrm{shots},\;
    \mathrm{basis},\;
    \mathrm{noise\_tag}
  \bigr),
\]
where $n$ is the number of qubits, \(\mathrm{shots}\) is the shot count used in the
experiment, \(\mathrm{basis}\) is the measurement basis (typically ``computational''),
and \(\mathrm{noise\_tag}\) is a high-level description of the noise regime
(e.g.\ ``ideal'', ``native'', ``mitigated'').

\paragraph{Witness profile.}
Given two labeled classes (e.g.\ honest vs.\ impostor), a \emph{witness} is a
scoring function $S$ that maps each observed event to a real score. Sweeping a
threshold $\tau$ yields a receiver-operating characteristic (ROC) curve
\[
  \bigl(\mathrm{FPR}(\tau),\,\mathrm{TPR}(\tau)\bigr)_{\tau},
\]
from which we extract summary metrics such as the area under the ROC curve (AUC)
and the true positive rate at a fixed false positive rate (e.g.\ 1\%).

We package the witness information into
\[
  \mathsf{Witness} \coloneqq
  \bigl(
    \mathrm{witness\_id},\;
    \mathrm{params},\;
    \mathrm{ROC},\;
    \mathrm{metrics}
  \bigr),
\]
where:
\begin{itemize}
  \item $\mathrm{witness\_id}$ names the witness family or implementation version,
  \item $\mathrm{params}$ records any hyperparameters,
  \item $\mathrm{ROC}$ optionally stores sampled points of the ROC curve
        (false positive rates, true positive rates, thresholds),
  \item $\mathrm{metrics}$ at minimum stores the AUC and TPR at 1\% FPR.
\end{itemize}

\paragraph{Provenance.}
Finally, a \emph{provenance descriptor} is a tuple
\[
  \mathsf{Prov} \coloneqq
  \bigl(
    \mathrm{runs\_paths},\;
    \mathrm{card\_hash},\;
    \mathrm{notes}
  \bigr),
\]
where $\mathrm{runs\_paths}$ is a list of underlying run-JSON paths (e.g.\
\texttt{runs/u31\_t2\_wrong\_key\_ibm\_torino.json}), $\mathrm{card\_hash}$ is a
cryptographic hash of the attestation card itself (defined below), and
$\mathrm{notes}$ is free-form context.

\section*{Attestation card and canonical hash}

A \emph{U58 attestation card} for one experiment is the tuple
\[
  \mathsf{Card} \coloneqq
  \bigl(
    \mathsf{Bundle}^{\mathrm{pub}},\;
    \mathsf{Test},\;
    \mathsf{Meas},\;
    \mathsf{Witness},\;
    \mathsf{Prov}
  \bigr),
\]
where $\mathsf{Bundle}^{\mathrm{pub}}$ is the public projection of the underlying
NVADE key bundle:
\begin{itemize}
  \item the public family $\mathsf{Family}$ (family id, series id, truncation, etc.),
  \item the channel spec $\mathsf{Chan}$ (backend kind, backend name, $n$, shots,
        noise tag, extra configuration),
  \item the E0 adversary access mode from $\mathsf{Threat}$,
  \item a key label identifying the secret key without revealing $\theta$.
\end{itemize}
The internal parameter $\theta$ itself is not exposed on the card.

In the implementation, attestation cards are represented as JSON objects. To make
cards tamper-evident we define a \emph{canonical hash} as follows. Let
\(\mathsf{Card}\) be represented as a JSON object, and let \(\mathsf{Card}'\) be
the same object with the field \(\mathrm{card\_hash}\) set to \texttt{null}. Let
\(\mathrm{JSON}(\mathsf{Card}')\) denote the canonical JSON string obtained by
serializing with sorted keys and fixed separators. The attestation hash is
\[
  \mathrm{card\_hash}(\mathsf{Card})
  \coloneqq \mathrm{SHA256}\bigl(\mathrm{JSON}(\mathsf{Card}')\bigr),
\]
where SHA-256 is the standard 256-bit secure hash function. The stored hash on
the card must agree with this value for the card to be considered valid.

\section*{Usage}

In the current NVADE--SEC pipeline, U58 attestation cards are produced from
U31-style run artifacts as follows:

\begin{itemize}
  \item A NVADE key bundle $\mathsf{Bundle}$ (E0) is fixed, specifying the family,
        backend and threat model.
  \item A U31 experiment is executed (e.g.\ trapdoor witness or wrong-key test),
        producing a run JSON with backend identifiers, shot counts, ROC data and
        summary metrics.
  \item The run JSON and $\mathsf{Bundle}$ are combined into a single
        $\mathsf{Card}$, filling the \(\mathsf{Test}\), \(\mathsf{Meas}\),
        \(\mathsf{Witness}\) and \(\mathsf{Prov}\) components.
  \item The canonical hash $\mathrm{card\_hash}(\mathsf{Card})$ is computed and
        written back to the card.
\end{itemize}

These attestation cards serve as the canonical evidence objects for NVADE--SEC:
they can be stored, signed, or published, and downstream analyses (including
noise-window advantage studies) operate on families of such cards.
\end{document}
